\section{Introduction}

The deployment of Cloud-Native applications (CNA), relies on placing a
collection of small, independent, loosely coupled and, ideally, self-managed
services into a cluster. This architectural style brings some advantages to the
applications. Due to the high decoupling of application's components, it is
possible to isolate failures to avoid the crash of the overall application.
Moreover, this architectural style allows a decoupled evolution of the
application by updating each component separately. However, the high decoupling
makes it difficult to manage the application, since there are many
distributed pieces of software that the engineers should \arsj{handle -- vague.
what is the problem, particularly one that you are focused on solving} during the
application execution.

To mitigate some of the challenges of CNA, the CNA services are bundled with
several runtime abstraction layers to facilitate execution or runtime
management. Some examples of these layers are application servers (e.g.,
OpenLiberty), runtime environments (e.g., Java Virtual Machine), hardware
isolation (e.g., containers and virtual machines), and cluster orchestrators
(e.g., Kubernetes).

Runtime abstractions layers are general purpose and have many parameters
(knobs), that engineers adjust to better satisfy the application. Many of these
knobs are related to data structures or infrastructure mechanisms, so their
adjustment affects performance. An problem with these knobs is the number of
possibilities that an engineer has to consider.

Every layer added to an application exponentially increases the number of
possible configurations that the engineer has available.  Making matters worse,
an optimal configuration for an application is mutable: it should change
according to the environment, such as fluctuation in the number of user requests,
saturation of the cluster, version of libraries being used, and so on.  To tune
tuning an application, the engineer has to evaluate many configurations to find
one that extracts the most application performance in a given situation.

To tackle this challenge, we propose \name, a mechanism to automatically
identify and apply optimal configurations of CNA regarding the environment where
the application is deployed. The rationale of this mechanism is find out and
continually update an optimal configuration that extracts the maximum
performance of an application and may reduce the application's resource
consumption. A new optimum configuration should be computed whenever the
application performance drops based on changes of the application workload over
time -- incoming requests, resources consumption, and application performance.

\todo{give a roadmap about what SmartTuning is}

\todo{how the paper is structured}


