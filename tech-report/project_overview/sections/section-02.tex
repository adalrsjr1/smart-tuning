\section{\name challenges}
\todo{this section needs beter structure. Challenges and go through them one by
one; bold each challenge you discuss}

\todo{add devops perspective to this section}

\todo{also add to the problem that ops teams now have to manage a variety of
software components for a variety of applications.  they don't have time to
focus on perfecting each one}

In this section we explain \name using a real application example and discuss
the main challenges of applying our approach in the real world.
AcmeAir\urlfoot{https://github.com/blueperf/acmeair-monolithic-java} is an
implementation of a fictitious airline developed in Java. The application was
built with the ability to scale to billions of web API calls per day. AcmeAir is
bundled up with several runtime layers, each one dealing with a specific aspect
of application execution (\fig{fig:acmeair-bundle}). The inner most layer is the
application server OpenLiberty which transparently handles connections to the
database, components integration, and monitors of the application. Next, as a
Java application, both AcmeAir and OpenLiberty runs in the JVM, which abstracts
the execution of code, and the management of memory and threads. All these
components are encapsulated into a Docker Container, which abstracts the file
system layer and several libraries of the OS. Finally, the container is
encapsulated into a Kubernetes Pod, which abstracts aspects regarding the
cluster management such as scaling out and roll out new version.

\begin{figure*}[htp]
    \centering
    \def\svgwidth{\textwidth}
    \scalebox{1.0}{\input{figs/acmeair-bundle.pdf_tex}}
    \caption{AcmeAir bundle.}
    \label{fig:acmeair-bundle}
\end{figure*}

Each of these layers exposes its own configuration interface with many options
that engineers use to set up the execution of the application. The combination
of configurations among interfaces affects the behavior of the application. For
example, the number of threads set in OpenLiberty's thread pool is directly
affected with the max heap size set in JVM, the number of cores and memory
available to the container, and the affinity set in a pod, so that Kubernetes
deploys the application into nodes with more or less co-located pods. Therefore,
a misconfiguration in one of these layers can drop the performance of the
application. Notice that for the simple knobs controlling thread pool size, the
number of possibilities may easily explode to hundreds of valid configurations,
making it unfeasible for engineers to try all, while looking for one that
maximizes the application's performance.

An application like AcmeAir has several aspects which engineers want to tune.
Moreover, some of these configurations have interdependencies,
for example, the number of max connections which should remain open in
OpenLiberty depends on the size of its thread pool. Or, the garbage collection
policy set depends on the heap size available in the JVM, and so on. It
increases even more the complexity to find out and try all valid configurations
of an application.

Another consideration when configuring an application is the impact of the
environment. An application running with the same configuration and the same
workload may have different performance due to other applications in the
cluster. For instance, in a given time AcmeAir is deployed with very few other
co-located applications.  However, in another time AcmeAir remains with the same
incoming rate and number of replicas, but the number of replicas of other
applications drastically increase.  Consequently, AcmeAir starts to suffer CPU
contention by other replicas co-located in the cluster node dropping its
performance.

Finally, modern CNA like AcmeAir are frequently evolving, and each step in their
evolution may require a different configuration, which adds a new level of
complexity in finding an optimal configuration. At the end of the day, it
becomes impossible for application engineers try all possible configurations for
each new scenario manually. Engineers to try a configuration which suits the
application in all possible scenarios.  However, they have no guarantee if this
configuration will be good enough for the application at all times, and a
careless set up eventually leads the application to a bad performance.

We propose \name to automatically support the search of good configurations.
\name identifies configurations based on analysis of the environment around the
application and the application itself, tuning the application to reach the best
possible performance. The concept of automatically tuning an application's
configuration is well known and has been successfully applied in the context of
machine learning~\cite{}, in Machine Learning domain named \emph{hyper
parameters optimization}.  However, applying in the same strategy of machine
learning in the context of CNAs is a naive solution.

As we mentioned, CNAs change behavior many times over time.  Moreover, a CNA
must satisfy many functional and non-functional requirements, which makes it a
``multi-purpose'' application.  Lastly, the lifespan of a CNA is
nondeterministic, it runs indefinitely since in most of the cases it provides a
service.

\todo{start by discussing the application of ML to your problem context --  On
the other hand}, machine learning applications are mostly defined to satisfy a
single functional requirement -- training a model to do the specific task --
with very few non-functional requirements -- fast training and high model
accuracy.  Besides, machine learning does not frequently evolve like CNAs, nor
suffer as many impacts from the environment. Therefore, a machine learning
configuration remains the same during all the time since it is deterministic,
differently of CNAs.


\todo{assumtion $\rightarrow$ next reaction?} Therefore, to apply the idea of
automatically finding a configuration to an application (or auto-tuning), we
assume that changes in the environment and of the application are periodic, and
these changes are observed thought application's workloads, so that \name can
identify patterns and associate them to configurations. Every new pattern that
\name observes from an application or the environment, which drops application
performance, trigger \name to figure out the best new configuration for this
pattern. Hence, \name can handle this pattern properly in the next time it comes
up.

Therefore of using \name, it may be unfeasible to find an optimal configuration
in a reasonable time.  \name should be used in collaboration with the
application engineers, so that they delimit the search space boundaries for
\name. Furthermore, we list in \crossref{sec:design.choices} other choices we
made to make it possible to apply auto-tuning in CNAs.
