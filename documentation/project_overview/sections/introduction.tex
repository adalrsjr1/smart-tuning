\section{Introduction}

The deployment of Cloud-Native applications (CNA), relies on placing a collection of small, independent, loosely
coupled and, ideally, self-managed services into a cluster. This architectural style brings reliability to the
application, due to decoupled failures be isolated to avoid the crash of the whole application, and facilitates a
decoupled evolution of theirs components. However, the high decoupling in this architectural style makes difficult the
application management, considering the large number of distributed pieces of software that engineers should handle
during the execution of the application.

Consequently, these services are mostly bundled up with several runtime abstraction layers to facilitate the their
execution or runtime management. Some examples of these layers are application servers (e.g., OpenLiberty), runtime
environment (e.g., Java Virtual Machine), hardware isolation (e.g., containers and virtual machines), and cluster
orchestrator (e.g., Kubernetes).

Runtime abstractions layers are general purpose with many parameters (knobs), that the application engineer adjusts to
better satisfy the use cases of the application. Many of these knobs are related to data structures or infrastructure
mechanisms, so their adjustment affects the application performance. An important issue regarding these knobs is the
colossal number of possibilities that an engineer has to consider to improve the application execution.

Every layer added to an application increases exponentially the number of possible configurations that the engineer has
available to experiment. Making matters worse, an optimal configuration for an application is mutable, it should
changes according to the environment, such as fluctuations of number of users requests, saturation of the cluster,
version of libraries being used, and so on. For tuning an application the engineer has to evaluate many different
configurations to find one that extracts the most of the application performance for a given situation.

To tackle this challenge, we are proposing \name, a mechanism to automatically identify and apply optimal
configurations onto CNA regarding the environment where they are deployed. The rational of this mechanism is, based on
changes of the application workload along the time (incoming requests and resources consumption), find out and
continually updates, a (quasi-) optimal configuration which extracts the maximum performance of the application and,
might, reduce its resource consumption.

