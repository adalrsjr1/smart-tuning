\section{Design Overview}

The \name operation unfolds into two main steps: to identify and group workloads of the application, and find out an
optimal configuration to a given workload. \fig{fig:design-overview} depicts these steps.

\begin{figure*}[htp]
    \centering
    \def\svgwidth{\textwidth}
    \scalebox{1.0}{\input{figs/design-overview.pdf_tex}}
    \caption{\name Overview.}
    \label{fig:design-overview}
\end{figure*}

We model an application as a black-box function $X: (c_k,\, t) \rightarrow w_t$, where $X$ maps a configuration $c_k$
in a time interval $t$ to an workload $w_t$.  A configuration is a set of knobs which the engineer can set in the
application's runtime layers. An workload, $w_{t} = (b_t \in B, s_t \in S, p_t \in P)$, models the behavior, the state,
and the performance of the application along a time interval $t$.

$B$ are all possible behaviors of application and $b_t$ models the application's behavior through the distribution of
urls hits against the application endpoints in the interval $t$, i.e., urls histogram. In turn, $S$ and $P$ are all
possible states and performance of an application, and $s_t$ and $p_t$ models respectively the resource consumption,
e.g., CPU and memory, and application performance of the application, e.g., throughput and latency in the same interval
$t$. Therefore, \name aims to improve the application $X$, $\text{\name} = \argmax_c X(c, t)$. The improvement of $X$
is given by a configuration $c^*$ which maximize the performance $p_t$ and might minimize the resource consumption
$s_t$ subject to an behavior $b_t$ in a time interval $t$, \eq{eq:optimization}.

\begin{equation}
  X(c^{*}, t) = \max{X(c, t)} =
  \begin{cases}
    \{s_t\, | \,s_t \in S \land \forall r_t \in S: s < r\} \\
    \{p_t\, | \,p_t \in P \land \forall q_t \in P: p > q\}
  \end{cases},\, \text{subject to}\, b_t
  \label{eq:optimization}
\end{equation}

Initially, the module named Classifier observes the workloads of the application under analysis. It groups the
workloads based on similarities among behavior and states, normalizing them into types, \eq{eq:classify}. As next step,
the Classifier learns when a workload type comes up from the application, associating an workload type to time
time-intervals while the application is running, \eq{eq:training}. Finally, it can forecast which workload type will
come up from the application for a given time interval $t$, \eq{eq:forecast}.

\begin{equation}
  K: w_t \rightarrow type_i
  \label{eq:classify}
\end{equation}

\begin{equation}
  T: (t,\, type_i) \cup L, L = \{ \forall t\,\exists \text{type}\, |\, t \rightarrow \text{type}_i \}
  \label{eq:training}
\end{equation}

\begin{equation}
  F: (t, \, L) \rightarrow type_i
  \label{eq:forecast}
\end{equation}

After \name has learned about the workload's application, it is time of the module named Tuning figures out the best
configuration for the application. The Tuning module uses the Classifier to forecast when and which will be the next
workload type that will come up so it can compute a new configuration. For each workload type, Tuning computes several
configurations trying to find out the one, $c^{*}$, which best improves the application. Then, \name updates the
learning model $L$, appending the optimal configuration of each type, \eq{eq:training-updated}.

\begin{equation}
  T': (\text{type}_i, c^{*} )\cup L', L' = \{ \forall\, \text{type}\, \exists\, c^{*}\, |\, \argmax_c X(c,t) = c^{*}\}
  \label{eq:training-updated}
\end{equation}

When Tuning applies a configuration to the application, \name observes differences on application workload's
performance $p$ or resource consumption $s$. Because the application is a black-box function, and thus unknown whether
it is differentiable, \name has to sequentially tries different configurations $c_k$ looking for one which leads the
application to its best -- high performance or low resources consumption, \fig{fig:tuning-overview}. To do so, Tuning
has to try different configurations for a same type of workload and maintain internally a history of the configurations
already tested against the application.  Tuning applies a configuration $c_k$ to the application and wait $t'$ for new
application's state and performance. The number of iterations to find an optimal configuration depends on the
application and which optimization technique \name uses.

\begin{figure*}[htp]
    \centering
    \def\svgwidth{\textwidth}
    \scalebox{1.0}{\input{figs/tuning-overview.pdf_tex}}
    \caption{K-th tuning iteration for workload of type X.}
    \label{fig:tuning-overview}
\end{figure*}

After $n,\, n \geq k$ iterations for workloads of a same type, \name learns which is the best configuration for that
type. The optimization is driven by the types of workloads, whenever a new type comes up, Tuning has to internally
changes its context to does not mess up with the computations made previously for the other types.  To guarantee that
each type has an optimal configuration, Tuning maintain an inner state for each type of workload under analysis.
\fig{fig:tuning-overview} depicts its behavior. When Classifier forecasts a new workload type from the application,
Tuning computes a new configuration and evaluates if it improves the application $X$.


