\section{\name Overview}

To compute configurations for applications, \name relies on three main phases:
workload classification, application tuning and tuning-workload prediction. Next
we overview the \name's design, listing step-by-step how to maintain an
application with good configurations, and highlighting the main phases of its
workflow.

\subsection{\name's algorithm}

 \begin{itemize}

   \item Stage 1a - observe workload

     \begin{itemize}

       \item watch incoming URLs

       \item build a internal representation that (histogram)

       \item watch resource consumption

       \item defines workload in terms of incoming URLs and resource consumption

       \item watch trends over time

     \end{itemize}

   \item Stage 1b - trend analysis of workload

     \begin{itemize}

       \item group the workloads in types

       \item use model techniques to categorize the workload (e.g., appear there
         are 3 different workloads at different times)

       \item use forecasting techniques to learn when a workload category arises

     \end{itemize}

   \item Stage 2 - compute configuration

      \begin{itemize}

        \item use hyper-parameter optimization techniques to pick a
          configuration candidate for tuning the application

        \item continually evaluates the performance of the application given a
          new configuration and a workload and drops the configuration if the
          performance reached is worst then a previously given a same context
          (i.e., workload type)

        \item contextualize the search of a optimal configuration with a
          workload type

        \item associates configurations and workloads types

      \end{itemize}

   \item Stage 3 - workload forecasting

      \begin{itemize}

        \item forecast next workload and configuration of the application

      \end{itemize}
 \end{itemize}

\subsection{Design Overview}

\todo{learning takes some time, how long we need to wait before to start
forecasting, we are going to forecast to avoid computing another a configuration
for an already known workload type --- describe this in this subsection or in
next}

The \name operation unfolds into three main steps: to identify and group
workloads of the application, learning an optimal configuration for a given
workload, and foresee next workload and configuration. \fig{fig:design-overview}
depicts these steps.

\begin{figure*}[htp]
    \centering
    \def\svgwidth{\textwidth}
    \scalebox{1.0}{\input{figs/design-overview.pdf_tex}}
    \caption{\name Overview.}
    \label{fig:design-overview}
\end{figure*}

We model an application as a black-box function $X: (c_k,\, t) \rightarrow w_t$,
where $X$ maps a configuration $c_k$ in a time $t$ to a workload $w_t$.  A
configuration is a set of knobs which the engineer can set in the application's
runtime layers. An workload, $w_{t} = (b_t \in B, s_t \in S, p_t \in P, d)$,
models respectively the behavior, the state, and the performance of the
application beginning in $t$ and with a duration $d$ set by application
engineer. The configuration $c_k$ affects the behavior and performance of the
application, and cannot affects affects the behavior of the application which
reflects the incoming URLs to the application.

$B$ are all possible behaviors of application and $b_t$ models the application's
behavior through the distribution of incoming URLs against the application
endpoints in the interval $(t, t+d)$, i.e., urls histogram. In turn, $S$ and $P$
are all possible states and performance of an application, and $s_t$ and $p_t$
models respectively the resource consumption, e.g., CPU and memory, and
application performance of the application, e.g., throughput and latency in the
same interval $(t, t+d)$. Therefore, \name aims to improve (max) the application
$X$, i.e., to find an optimal configuration $c^{*}_{k}$ which maximizes the
performance $p_t$ and might minimize the resources consumption $s_t$ both
subject to a behavior $b_t$ in a time interval $(t, t+d)$. \eq{eq:optimization}

\begin{equation}
  X(c^{*}, t) = \max{X} =
  \begin{cases}
    \min(S) \\
    \max(P)
    %\{s_t\, | \,s_t \in S \land \forall r_t \in S: s < r\} \\
    %\{p_t\, | \,p_t \in P \land \forall q_t \in P: p > q\}
  \end{cases},\, \text{subject to}\, b_t
  \label{eq:optimization}
\end{equation}

Initially, the module named Classifier observes the workloads of the application
under analysis. It groups the workloads based on similarities of their behavior
or states, normalizing them into types, \eq{eq:classify}. As next step, the
Classifier uses techniques of time-series analysis to learn when a workload type
comes up from the application. It associates the workload type to a time step
while the application is running, \eq{eq:training}.  Hence, Classifier can
forecast which workload type will come up from the application in a given time
$t$, \eq{eq:forecast}.

\begin{equation}
  K: w_t \rightarrow type_i
  \label{eq:classify}
\end{equation}

\begin{equation}
  T: (t,\, type_i) \cup L, L = \{ \forall t\,\exists \text{type}_i\, |\, t \rightarrow \text{type}_i \}
  \label{eq:training}
\end{equation}

\begin{equation}
  F: (t, \, L) \rightarrow type_i
  \label{eq:forecast}
\end{equation}

After \name has learned about the workload's application, \arsj{not clear how
\name is learning about the workloads's application at this point -- so far in
the flow it's just looking at incoming workload} it is time of the module named
Tuning figures out the best configuration for the application. The Tuning module
uses the Classifier to forecast when and which will be the next workload of the
application so it can compute a new configuration. For each workload type,
Tuning computes several configurations trying to find out the one, $c^{*}$,
which best improves the application. Then, \name updates the learning model $L$,
appending the optimal configuration of each type, \eq{eq:training-updated}.

\begin{equation}
  T': (\text{type}_i, c^{*} )\cup L', L' = \{ \forall\, \text{type}\, \exists\, c^{*}\, |\, \argmax_c X(c,t) = c^{*}\}
  \label{eq:training-updated}
\end{equation}

When Tuning applies a configuration to the application, \name observes
differences on application workload's performance $p$ or resource consumption
$s$. Because the application is a black-box function, and thus unknown whether
it is differentiable, \name has to sequentially tries different configurations
$c_k$ looking for one which leads the application to its best -- high
performance or low resources consumption, \fig{fig:tuning-overview}. To do so,
Tuning try different configurations for a same type of workload and maintain
internally a history of the configurations already tested against the
application.  Tuning applies a configuration $c_k$ to the application and wait
$d$ for new application's state and performance. The new configuration feedbacks
the Classifier, and updates its learning model if the performance of the
application is improved. The number of iterations to find an optimal
configuration depends on the application and which optimization technique \name
uses.

\begin{figure*}[htp]
    \centering
    \def\svgwidth{\textwidth}
    \scalebox{1.0}{\input{figs/tuning-overview.pdf_tex}}
    \caption{K-th tuning iteration for workload of type X.}
    \label{fig:tuning-overview}
\end{figure*}

After $n,\, n \geq k$ iterations with a same type of workloads, \name learns
which is the best configuration for that type. The optimization is driven by the
types of workloads, whenever a new type comes up to Tuning, it has to internally
changes its context to does not mess up with the computations made previously
for the other types.  To guarantee that each type has an optimal configuration,
Tuning maintain an inner state for each type of workload previously analyzed.
\fig{fig:tuning-overview} depicts its behavior. When Classifier forecasts a new
workload type from the application, Tuning computes a new configuration and
evaluates if it improves the application $X$.

\subsection{\name phases}

\todo{write a quick summary here}

\subsubsection{Workload classification}

The main objective of \name is to associate an application's optimal
configuration to a workload. To do so, first \name has to characterize the
workloads. Along the time, several different workloads come up from the
application. To reduce the scope and the number of configurations to be
computed, \name characterizes the workloads into groups.

\name characterizes a workload in a multidimensional histogram-ish structure
constituted with the workload's behavior ($b$), state ($s$), and performance
($p$) of the application in a given time interval. The application engineer
should set in \name the length $d$ of the time interval which bounds every
workload. Hence, periodically, \name samples all suitable data from the
application within the interval $(t_i, t_{i+d})$ to characterize the workloads.
For instance, \name samples data regarding AcmeAir from Prometheus periodically,
in intervals of $d = 6$ hours to build up the workloads.
\fig{fig:workload-classification} depicts the overview of workload
classification.

\begin{figure*}[htp]
    \centering
    \def\svgwidth{\textwidth}
    \scalebox{1.0}{\input{figs/workload-classification-overview.pdf_tex}}
    \caption{Workload classification.}
    \label{fig:workload-classification}
\end{figure*}

The behavior of the workload is given by the incoming URLs to the application.
The distribution and frequency of these URLs along the time interval model the
shape and intensity of the workload behavior histogram respectively,
\fig{fig:histogram}.  The histogram is enhanced with state and performance
dimensions. The dimension state add an information about the resource
consumption of the application in the time interval. The resource consumption,
usually CPU, memory, and I/O components, is measured and the average, standard
deviation, and number of samples of each metric are kept in the histogram
structure.  Moreover, the average and standard deviation, and number of samples
of different performance metrics of the application, such as throughput and
latency, are kept in the histogram as well.

\begin{figure*}[htp]
  \centering
  \def\svcwidth{\textwidth}
  \scalebox{1.0}{\input{figs/histogram.pdf_tex}}
  \caption{Histogram behavior.}
  \label{fig:histogram}
\end{figure*}

Every workload that \name samples are grouped into types according to their
characteristics, reducing the number of configurations to be computed
afterwards. The workloads are compared and grouped according their similarities.
Finally, \name stores each workload type with a list of times when its comes up
for forecasting purposes.

The \name's Classifier groups the workloads by their behaviors and states. It
uses K-Nearest Neighbor (KNN)~\cite{?} with the number of groups $k$ set by the
application engineer. The core of KNN relies on the method to compute the
similarity between the elements being classified. Our approach considers the
behavior and state of workloads to calculate their proximity.

\todo{should K-nearest neighbors be static (the engineer set K) or dynamic --
\name defines how many clusters based on a given threshold set by engineer?}

The behavior of two workloads are compared using Hellinger distance~\cite{?}, an
statistic method to quantify the similarity, between 0 (equals) and 1
(different), of two probabilities distributions. Since Hellinger distance works
with probabilities distributions, the histograms should also be normalized
between 0-1.

To avoid miscalculations and to preserve the semantics of all histograms, the
classifier normalizes two behaviors by using MinMax~\cite{?}, setting min and
max as the smallest and largest values of both histograms under analysis, e.g.,
$h_1 = (3, 2, 5)$ and $h_2 = (2, 9, 5)$, $\text{min} = 2$ and $\text{max} = 9$.
Therefore, $h'_1 = (0.1, 0.0, 0.3)$ and $h'_2 = (0.0, 0.7, 0.3)$. Behaviors with
same shape and different intensities (height), means that the highest behavior
consumes more resources, since workload height is consequence of the number of
incoming URLs processed by application.

However, not always the resource consumption is related only to the quantity of
incoming requests. In some cases, a low resource consumption is consequence of
resources contention caused by another application which is co-located into a
same node of the first. Or, the resource consumption changes because the
application's replica under monitoring is running in a node with more or less
resources available them previously. For a few URL incoming rate, Classifier
cannot identify the variations on resources consumption by only analyzing the
workload behavior.

% https://people.richland.edu/james/lecture/m170/ch08-int.html
Therefore, Classifier uses Z-Score~\cite{?} to compare the state of two
workloads.  Given two workloads, for each their components $(a_0, a_1, \cdots,
a_n)$, e.g., CPU, memory, and I/O, in their states $s_i$, Classifier checks if
all means of each state component are significantly equals. For values of
$\text{Z-Score} \leq 1.960$, we can assume they are significantly equals, and
$\text{Z-Score} \geq 3$ they are significantly different.

The behavior and state of a workload are orthogonal concepts. Then, after the
Classifier calculates the similarity of both individually for two different
workloads,  it is necessary to bring these concepts to a same universe. Hence,
Classifier can calculate the real similarity of two workloads.  To do so,
Classifier assumes that the state of a workload is a vector $\hat{s}$ and
calculates its magnitude, $||\hat{s}||_2$ and direction $\theta \hat{s}$, both
from origin $(0,0)$.

The decomposition of $\hat{s}$ tells to Classifier the amount of each resource
is being used (magnitude) and discern the semantics of each component in the
vector (direction). For example, a resource consumption of 100 milicores and
1000MB has a different semantics of 1000 milicores and 100 MB, and consequently
the application may need different configurations to handle it. In both cases,
the magnitude are equal ($\sim 1005$), and their directions are different ($\sim
1.47\,\text{rad}$ and $\sim 0.09\,\text{rad}$ respectively).

Then, Classifier can transform the vector state $\hat{s}$ in in a scalar $s'$ by
calculating their distance on $\mathbb{R}^2$, \eq{eq:distance1}.

\begin{equation}
  s' = \sqrt{\text{magnitude}^2 + \text{direction}^2}
  \label{eq:distance1}
\end{equation}

Finally, with both behavior and state being scalars, Classifier can transform
both orthogonal concepts in a single metric $\text{distance}$,
\eq{eq:distance2}, used by KNN for grouping the workloads.

\begin{equation}
  \text{distance} = \sqrt{b^2 + s'^2}
  \label{eq:distance2}
\end{equation}

After the Classifier groups the workloads, as depicted in
\fig{fig:workload-classification}, it also associate to each group the time of
their fundamental workloads occurred. Moreover, it computes the average of application
performance and resource consumption and keeps a structure for each type as following:

\begin{quote}
  \centering
  ($\text{type}_k$, average of performance, list of time intervals)
\end{quote}

This structure is the core of the next phases of the \name. Tuning phase uses
this structure to calculate the improvement on the application when it apply a
configuration (it aims to minimize the resource consumption and maximize the
application performance). The Tuning-workload prediction uses the structure to
for modeling when a given type of workload arises in the application.

\subsubsection{Application tuning}

\todo{next week}

\subsubsection{Tuning-workload prediction}

\todo{next week}
