\section{Design choices} \label{sec:design.choices}

\arsj{missing something describing gradual changes, fast change and memory
leaks}

We have made some assumptions to design \name. Following a non-definitive list
of them.

\begin{itemize}

\item An application starts with a default configuration initially computed on
  its test phase. We assume that the optimal configuration found during the
    tests are good enough to give the application a better initial configuration
    than a set by engineers manually.

\item An application is an periodic function. We assume that if an application
  is observed long enough, some patterns on its behaviour and resources usage
    appears, so that \name can learns and use these patterns to improve how
    optimal configurations are computed and applied on applications.

\item The application is a black-box function. Despite our assumption of the
  application be a periodic function (and then differentiable), we cannot found
    its max and min analytically. It is impossible to know \textit{a priori} the
    frequency of the application's patterns neither the rules that define the
    slopes of their performance and resource usage along time. Therefore, we
    assume that the best way to find an optimal configuration is using a sort of
    sequential optimization method.

\item \name observes the patterns that come up from different aspects of the
  application, such resources consumption and its inner behavior. We consider
    that all applications are properly instrumented, and all metrics necessary
    are exposed by the application at tuning phase.

\item We are assuming that the inner behavior of the application can be modeled
  as a histogram of application's urls reached at runtime. Besides, we consider
    that the inner behavior is responsible to make changes on resources usages
    and performance of the application, i.e., different shapes of the histograms
    result on a different resource consumption and application performance.
    \arsj{this doesn't sound quite right -- we're assuming that the workload can
    be modeled that way, but not that the application itself can be}

\item Two equals workflow, i.e., both have same shape and intensity of for every
  buckets, may result in different resources consumption and performance of the
    application. So, we are assuming that histograms cannot suppress the need of
    resource consumption metrics.

\item Patterns that arise along the time from the application and can be grouped
  given a threshold. We suspect that patterns come up from the application
    frequently, but it may exist a slight differences among them. Besides, we
    assume that theses differences can be parametrized so that \name can group
    these patterns into different types.  \arsj{this one wasn't clear -- need to
    give example of what you mean}

\item We assume that changes on the patterns that come up from the application
  means the application has new non-functional requirements. So, for each new
    workload type a new configuration should be set to the application to get
    its best on performance and resources consumption. \arsj{this one wasn't as
    clear -- need to give example of what you mean}

\item We expect the application remains that same, i.e., without new
  deployments, long enough for \name computes a optimal configuration and the
    application still take advantage of the new configuration for a significant
    amount of time.

\item The application can be modeled by observing a reduced scope. We assume
  that the incoming traffic of the application is the only input necessary to
    provide changes on the application performance. Despite other variations on
    the environment may reflect on application performance, we are considering
    the environment stable. Hence, we can model the application behavior by
    measuring the only part incoming traffic, so that we can generalize the
    patterns of the whole application by observing the patterns that arise from
    a single replica, for instance. \arsj{memory leaks or I/O contention from
    other things sharing disks, or other things sharing services the app is
    dependent on may have impact}

  \item \arsj{engineers should set the length of the workload time interval --
    in future work this interval may be updated/set automatically}

\end{itemize}

