\section{Assumptions}

We have made some assumptions to design \name. Following a non-definitive list of them.

\begin{itemize}

\item An application starts with a default configuration initially computed on its test phase. We assuming that the
optimal configuration found during the tests are good enough to give the application a better initial configuration
than a set by engineers manually.

\item An application is an harmonic-ish function. We assuming that if an application is observed long enough, some
patterns on its behaviour and resources usage appears, so that \name can learns and use it to improve how optimal
configurations are computed and applied on applications.

\item The application is a black-box function. Despite our assumption of the application be a sort of harmonic function
(and then differentiable), we cannot found its max and min analytically. It is impossible to know \textit{a priori} the
frequency of the application's patterns neither the rules that define the slopes of their performance and resource
usage along time. Therefore, we assume that best way to find an optimal configuration is using a sort of sequential
optimization method.

\item \name observes the patterns that come up from different aspects of the application, such resources consumption
and inner behavior. We are assuming that all applications which \name observes are properly instrumented, and all
metrics necessary are exposed by the application at tuning phase.

\item We are assuming that the inner behavior of the application can be modeled as a histogram of which urls of an
application are reached at runtime. Besides, we assuming that the inner behavior is responsible to make changes on
resources usages and performance of the application, i.e., different shapes of the histograms result on a different
resource consumption and application performance.

\item Two equals workflow, i.e., both have same shape and intensity of all buckets, may results in different resources
consumption and performance of the application. So, we are assuming that histograms cannot suppress the need of
resource consumption metrics.

\item Along the time, my patterns arise from the application and they can be grouped given a threshold. We are assuming
that frequently patterns come up from the application, but they are not equals and may exists slight differences among
them. Besides, we assuming that theses differences can be parametrized so that \name can groups these different
patterns.

\item We are assuming that when a different type of pattern comes up from the application means that it has new
non-functional requirements. So, for each new workload type a new configuration should be set to the application to
get its best on performance and resources consumption.

\item We are assuming that the application remains that same, i.e., without new deployments, long enough so an optimal
configuration be properly computed and the application still take advantage of the new configuration for a significant
amount of time. These times are related to each application.

\item The application can be modeled by observing a reduced scope. We are assuming that the incoming traffic of the
application is the only input necessary to provides changes on the application performance. We also assuming that we
can model the application behavior by measuring the only part incoming traffic, so that we can generalize the patterns
of the application by observing the patters that arise from a single replica, for instance.

\end{itemize}

